\documentclass[10pt,a4paper,]{article}
\usepackage{float}
  \usepackage[]{mathpazo}
  \usepackage{setspace}

%\usepackage[T1]{fontenc} %RK added
\usepackage{lmodern} %RK added

\usepackage{amssymb,amsmath}
\usepackage{ifxetex,ifluatex}
\usepackage{fixltx2e} % provides \textsubscript
\ifnum 0\ifxetex 1\fi\ifluatex 1\fi=0 % if pdftex
\usepackage[T1]{fontenc}
\usepackage[utf8]{inputenc}
\usepackage{csquotes}
  \else % if luatex or xelatex
  \usepackage{unicode-math}
  \defaultfontfeatures{Ligatures=TeX,Scale=MatchLowercase}
              \fi
% use upquote if available, for straight quotes in verbatim environments
\IfFileExists{upquote.sty}{\usepackage{upquote}}{}
% use microtype if available
\IfFileExists{microtype.sty}{%
  \usepackage[]{microtype}
  \UseMicrotypeSet[protrusion]{basicmath} % disable protrusion for tt fonts
}{}
\PassOptionsToPackage{hyphens}{url} % url is loaded by hyperref
\usepackage{fancybox}
  \usepackage[unicode=true]{hyperref}
  \hypersetup{
    pdfauthor={Ryan N. Kinzer (Research Division, Department of
Fisheries Resources Management)},
          pdftitle={Survival Advantage for Natural-origin Summer Parr Reared in a Hatchery Environment},
            %                pdfborder={0 0 0},
          breaklinks=true}
                
\urlstyle{same}  % don't use monospace font for urls
\usepackage[margin=1in]{geometry}
\usepackage[style=authoryear-comp,backend=biber, natbib=true,]{biblatex}
\usepackage{longtable,booktabs}
% Fix footnotes in tables (requires footnote package)
\IfFileExists{footnote.sty}{\usepackage{footnote}\makesavenoteenv{long table}}{}
\IfFileExists{parskip.sty}{%
\usepackage{parskip}
}{% else
\setlength{\parindent}{0pt}
\setlength{\parskip}{6pt plus 2pt minus 1pt}
}
\setlength{\emergencystretch}{3em}  % prevent overfull lines
\providecommand{\tightlist}{%
  \setlength{\itemsep}{0pt}\setlength{\parskip}{0pt}}
\setcounter{secnumdepth}{0}

% set default figure placement to htbp
\makeatletter
\def\fps@figure{htbp}
\makeatother

\title{Survival Advantage for Natural-origin Summer Parr Reared in a
Hatchery Environment \vspace{0.5cm}}

\author{\Large \bf{Ryan N.
Kinzer}\vspace{0.05in} \newline\normalsize\emph{Research Division,
Department of Fisheries Resources Management}  }

%% MONASH STUFF

%% CAPTIONS
\RequirePackage{caption}
\DeclareCaptionStyle{italic}[justification=centering]
 {labelfont={bf},textfont={it},labelsep=colon}
\captionsetup[figure]{style=italic,format=hang,singlelinecheck=true}
\captionsetup[table]{style=italic,format=hang,singlelinecheck=true}

%% FONT
\usepackage{bm,url}

%% HEADERS AND FOOTERS
\RequirePackage{fancyhdr}
\pagestyle{fancy}
\lfoot{}\cfoot{}\rfoot{}
\lhead{\textsf{Summer Parr}} % was title
\rhead{\textsf{\thepage}}
\setlength{\headheight}{15pt}
\renewcommand{\headrulewidth}{0.4pt}
\fancypagestyle{plain}{%
\fancyhf{} % clear all header and footer fields
\fancyfoot[C]{\sffamily\thepage} % except the center
\renewcommand{\headrulewidth}{0pt}
\renewcommand{\footrulewidth}{0pt}}

%% MATHS
\RequirePackage{bm,amsmath}
\allowdisplaybreaks

%% GRAPHICS
\RequirePackage{graphicx}
\setcounter{topnumber}{2}
\setcounter{bottomnumber}{2}
\setcounter{totalnumber}{4}
\renewcommand{\topfraction}{0.85}
\renewcommand{\bottomfraction}{0.85}
\renewcommand{\textfraction}{0.15}
\renewcommand{\floatpagefraction}{0.8}

%\RequirePackage[section]{placeins}

%% SECTION TITLES
\RequirePackage[compact,sf,bf]{titlesec}
\titleformat{\section}[block]
  {\fontsize{15}{17}\bfseries\sffamily}
  {\thesection}
  {0.4em}{}
\titleformat{\subsection}[block]
  {\fontsize{12}{14}\bfseries\sffamily}
  {\thesubsection}
  {0.4em}{}
\titlespacing{\section}{0pt}{*3}{*1}
\titlespacing{\subsection}{0pt}{*1}{*0.5}

%% LINE AND PAGE BREAKING
\sloppy
\raggedbottom
\usepackage[bottom]{footmisc}
\clubpenalty = 10000
\widowpenalty = 10000
\brokenpenalty = 10000
\RequirePackage{microtype}

%% HYPERLINKS
\RequirePackage{xcolor} % Needed for links
\definecolor{darkblue}{rgb}{0,0,.6}
\RequirePackage{url}

\makeatletter
\@ifpackageloaded{hyperref}{}{\RequirePackage{hyperref}}
\makeatother
\hypersetup{
     citecolor=0 0 0,
     breaklinks=true,
     bookmarksopen=true,
     bookmarksnumbered=true,
     linkcolor=darkblue,
     urlcolor=blue,
     citecolor=darkblue,
     colorlinks=true}

\usepackage[showonlyrefs]{mathtools}

%% BIBLIOGRAPHY

\makeatletter
\@ifpackageloaded{biblatex}{}{\usepackage[style=authoryear-comp, backend=biber, natbib=true]{biblatex}}
\makeatother
\ExecuteBibliographyOptions{bibencoding=utf8,minnames=1,maxnames=3, maxbibnames=99,dashed=false,terseinits=true,giveninits=true,uniquename=false,uniquelist=false,doi=false, isbn=false,url=true,sortcites=false}
\DeclareFieldFormat{url}{\texttt{\url{#1}}}
\DeclareFieldFormat[article]{pages}{#1}
\DeclareFieldFormat[inproceedings]{pages}{\lowercase{pp.}#1}
\DeclareFieldFormat[incollection]{pages}{\lowercase{pp.}#1}
\DeclareFieldFormat[article]{volume}{\mkbibbold{#1}}
\DeclareFieldFormat[article]{number}{\mkbibparens{#1}}
\DeclareFieldFormat[article]{title}{\MakeCapital{#1}}
\DeclareFieldFormat[article]{url}{}
%\DeclareFieldFormat[book]{url}{}
%\DeclareFieldFormat[inbook]{url}{}
%\DeclareFieldFormat[incollection]{url}{}
%\DeclareFieldFormat[inproceedings]{url}{}
\DeclareFieldFormat[inproceedings]{title}{#1}
\DeclareFieldFormat{shorthandwidth}{#1}
%\DeclareFieldFormat{extrayear}{}
% No dot before number of articles
\usepackage{xpatch}
\xpatchbibmacro{volume+number+eid}{\setunit*{\adddot}}{}{}{}
% Remove In: for an article.
\renewbibmacro{in:}{%
  \ifentrytype{article}{}{%
  \printtext{\bibstring{in}\intitlepunct}}}
\AtEveryBibitem{\clearfield{month}}
\AtEveryCitekey{\clearfield{month}}
\makeatletter
\DeclareDelimFormat[cbx@textcite]{nameyeardelim}{\addspace}
\makeatother
\renewcommand*{\finalnamedelim}{%
  %\ifnumgreater{\value{liststop}}{2}{\finalandcomma}{}% there really should be no funny Oxford comma business here
  \addspace\&\space}

%%% Change title format
\usepackage{color,titling,framed}
\usepackage[absolute,overlay]{textpos}
\setlength{\TPHorizModule}{1cm}
\setlength{\TPVertModule}{1cm}

\pretitle{%
%
\begin{textblock}{4}(2,0.8)\includegraphics[height=2.0cm]{NPT.png}\end{textblock}%
\begin{textblock}{4}(17,0.8)\includegraphics[height=2.0cm]{DFRM.png}\end{textblock}%
%\begin{textblock}{4}(16.5,0.8)\includegraphics[trim=53 0 0 0, clip=true,height=1.5cm]{NPT.png}\end{textblock}%
\vspace*{-1.0cm}%-0.6

\LARGE\bfseries}
\posttitle{\vspace*{0.3cm}\par}
\preauthor{\large}
\postauthor{\hfill}
\predate{\hfill\small}
\postdate{}

\raggedbottom

\usepackage[australian]{babel}
\date{June 07, 2024}



\begin{document}


  \vspace*{0cm}
  \definecolor{shadecolor}{RGB}{255,255,255} %210
  \begin{snugshade}%\sffamily
  \maketitle
  \end{snugshade}\vspace*{0.5cm}
  \definecolor{shadecolor}{RGB}{248,248,248} %248

\hbox{\vrule height .2pt width 16cm}

%\hbox{\vrule height .2pt width 16cm} % RK added

\setstretch{1.2}


\hypertarget{background}{%
\section{Background}\label{background}}

A proposed conservation strategy for sp/sm Chinook salmon is to collect
natural-origin parr during the summer months and rear them in a hatchery
environment until release as smolts the following spring season. The
assumed advantage of this action is the removal of natural rearing
mortality during the fall and overwinter period. The aim of this
exercise is to calculate the expected survival advantage for hatchery
reared natural-origin parr compared to their natural rearing
counterparts.

\hypertarget{methods}{%
\section{Methods}\label{methods}}

First, we assume the number of adults surviving to the tributary from
summer parr is the product of three life-stage survival probabilities.
The first stage is summer parr surviving to smolt leaving the tributary,
in other words, overwinter survival (\(\phi_{ow}\)). The second stage
represents smolt migration survival for fish traveling from the
tributary to Lower Granite Dam (\(\phi_s\)). The third stage is commonly
known as a smolt-to-adult return rate (SAR), and here, represents the
survival of smolts at Lower Granite Dam (LWG) returning as adults to the
tributary (\(\phi_{SAR}\)). The product of the equation then becomes the
survival probability of summer parr returning as adults
(\(\phi_{parr}\)).

\[
\phi_{parr} = \phi_{ow}*\phi_s*\phi_{SAR}
\] The relative survival advantage (\(RSA\)) of rearing parr in the
hatchery environment then becomes the ratio of parr survival
probabilities for fish reared in the hatchery environment
(\(\phi^{H}_{parr}\)) to fish reared naturally in the stream
(\(\phi^{N}_{parr}\)). This concept is similar to the more commonly used
relative reproductive success metric often produced from genetic
parentage analyses.

\[
RSA = \frac{\phi^{H}_{parr}}{\phi^{N}_{parr}}
\] To calculate the ratio we can substitute estimates of the three
life-stage survivals for hatchery and natural rearing using our existing
data, with one exception. We don't directly monitor natural rearing
overwinter survival (i.e., from parr to smolt within the tributary). To
indirectly estimate this stage we can use the estimates of natural parr
and natural smolts surviving to LWG (\(\phi^N_{p}\), \(\phi^N_{s}\))
from our juvenile PIT-tagging efforts as follows,

\[
\phi^N_w = \frac{\phi^N_{p}}{\phi^N_{s}}.
\]

Then, after substituting the above calculations and simplifing the full
expanded \(RSA\) equation becomes,

\[
RSA = \frac{\phi^H_{ow}*\phi^H_s*\phi^H_{SAR}}{\phi^N_p*\phi^N_{SAR}}.
\] We now can go an additional step in determining the success of the
proposed conservation strategy by solving the above equation for
\(\phi^H_{ow}\). Wee can then set \(RSA\) to the management desired
minimum that would indicate success of the program and calculate the
lowest acceptable hatchery rearing survival that would still yield an
RSA above the desired minimum.

\[
MIN(\phi^H_{ow}) =\frac{MIN(RSA)*\phi^N_p*\phi^N_{SAR}}{\phi^H_s*\phi^H_{SAR}}
\]

\hypertarget{example}{%
\section{Example}\label{example}}

If we assume the parameter values shown in Table 1 the calculated
\(RSA\) is 1.92. Thus, if 5,000 natural parr were collected and reared
in the hatchery environment we would expect approximately 1.92 times
more adults to return than if the parr reared naturally.

Additionally, if we set \(RSA = 1.5\) as a threshold to success,
hatchery overwinter survival must be greater than 0.625 to deem the
program successful.

\begin{longtable}[]{@{}llr@{}}
\caption{Example parameter values to calculate the relative survival
advantage (RSA).}\tabularnewline
\toprule\noalign{}
Rearing & Parameter & Value \\
\midrule\noalign{}
\endfirsthead
\toprule\noalign{}
Rearing & Parameter & Value \\
\midrule\noalign{}
\endhead
\bottomrule\noalign{}
\endlastfoot
Hatchery & \(\phi^H_{ow}\) & 0.800 \\
Hatchery & \(\phi^H_{s}\) & 0.600 \\
Hatchery & \(\phi^H_{SAR}\) & 0.008 \\
Natural & \(\phi^N_{p}\) & 0.200 \\
Natural & \(\phi^N_{SAR}\) & 0.010 \\
\end{longtable}

\printbibliography

\end{document}